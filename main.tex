\documentclass{beamer}
\usepackage[utf8]{inputenc}
\usepackage{natbib}
\usepackage{graphicx}
\hypersetup{
	colorlinks=true,
	linkcolor=blue,
	filecolor=magenta}

\begin{document}

\begin{frame}{AMAZON}
	\includegraphics[height= 9 cm]{/home/arwa/Documents/Presentation/amazon}

\end{frame}


\begin{frame}{Team Members}
    
\begin{itemize}
    \item Muhammad El-Sayed (Team Leader)
    \pause
    \item Arwa Sharaqi
    \pause
    \item Nada El-Attafy
    \pause
    \item Ahmad Abdullah
    \pause
    \item Mohammed Ouda
    
\end{itemize}
\end{frame}

\begin{frame}{Does Amazon contribute to / support OS ?}
	\begin{itemize}
		
	\item Amazon avoids contributing to open source because it wants to hold it's competitive edge, But without\\ \huge open source \normalsize Amazon wouldn't be where it is today!
	
	\item The company, both the retail and the cloud parts of it, is completely built on open source technologies.
	
\end{itemize}
\end{frame}

	\begin{frame}{Does Amazon contribute to / support OS ?}
		\begin{itemize}

	\item Amazon has open sourced some really cool things lately :
	\\
	Example - \href{https://github.com/amznlabs/ion-java}{Ion Java}.
		
		\end{itemize}
	\end{frame}
	
	
	\begin{frame}{Does Amazon contribute to / support OS ?}
		
		\huge Amazon DSSTNE
		\\ \normalsize- Deep Scalable Sparse Tensor Network Engine -
		
		\begin{itemize}	
			\item What is DSSTNE ?
			\item Amazon engineers built DSSTNE to solve deep learning problems at Amazon's scale.
			\item It is built for production deployment of real-world deep learning applications, emphasizing speed and scale over experimental flexibility.
			
 		\end{itemize}
	\end{frame}

\begin{frame}{Does Amazon uses OS ?}
	\centering "Amazon \huge{cannot exist} \normalsize without open source".
	\\ - said one former Amazonian.
\end{frame}

\begin{frame}{AWS – Amazon’s Major Product}
\begin{itemize}
	
	\item Amazon Web Services (AWS) Definition.
	\item Most prominent services :
	\\ - EC2, which offers virtual servers.
	\\- S3, which offers storage capacity.
	\item With AWS you can ...
	
\end{itemize}
\end{frame}

\begin{frame}{Do you consider amazon a pure OS ?}
	\centering "Within Amazon it was well known that [CEO] Jeff Bezos didn't think that Amazon would gain from participating in open source except in very limited ways at the fringes of its tech".
	\\"Amazon really kept its code \huge{closed}\normalsize."
	\\ \normalsize - Said one ex-Amazonian. 
\end{frame}


\begin{frame}{What are the benefits of supporting OS for Amazon and the community ?}
	\begin{itemize}
		
		\item (FOSS) holds numerous other compelling advantages for Hugh Tech Companies businesses.
		\item Examples : Security, Customizability.
		
	\end{itemize} 
\end{frame}


\begin{frame}{Do you consider initiating your start up in OS model ?}
	\centering 
	Well,  I’m not saying , of course, that my business should necessarily use open source software for everything. 
	But with all the many benefits it holds, I'd be remiss not to consider it seriously.
	\\ - Personal Opinion.
\end{frame}

\begin{frame}{What is the OS license that Amazon uses ?}
	
	\centering \huge \href{https://aws.amazon.com/asl/}{Amazon Software License.}
\end{frame}


\begin{frame}
\centering \huge {Thank you for your time.}
\end{frame}

\begin{frame}{References}
	
	- \url{http://www.theregister.co.uk/2014/01/22/amazon_open_source_investigation/}
	
	- \url{http://www.pcworld.com/article/209891/10_reasons_open_source_is_good_for_business.html}
	
	- \url{https://www.quora.com/Has-Amazon-contributed-to-the-open-source-community}
	
	- \url{https://aws.amazon.com/asl/}
	
	- \url{https://issues.apache.org/jira/browse/LEGAL-198}
	
\end{frame}

\end{document}
